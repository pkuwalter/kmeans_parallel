\section{Parallel k-means clustering}
An obvious way to parallelize the algorithm is to parallelize the
parts for membership assignment and recomputation on new centroids.

There are generally two ways to compute distance between a point $p(i)$
and centroid $c(j)$. Suppose $p(i) = (p_{i1}, p_{i2}, ..., p_{id})$ 
and $c(j) = (c_{j1}, c_{j2}, ..., c_{jd})$, the squared Euclidean distance 
between point p(i) and centroid c(j) can be computed as: 
\begin{align*}
  dist^2(p(i),c(j)) &= (\vec{p_i} - \vec{c_i})^2 \\
                 &= \sum\limits_{k}^d (p_{ik} - c_{jk})^2 \numberthis \label{e1}\\
             &= \norm{\vec{p_i}}^2 - 2 \vec{p_i} \cdot \vec{c_i^T} + \norm{\vec{c_i}}^2 \\
             &= \sum\limits_{k}^d p_{ik}^2 - 2 \sum\limits_{k}^d p_{ik}*c_{jk} + \sum\limits_{k}^d c_{jk}^2 \numberthis \label{e2}
\end{align*}

This leads to two categories of parallel implementation, respectively using 
(\ref{e1}) and (\ref{e2}). Their computational complexity under the big O notation
is the same. But theoretically their number of floating point computations has a 
constant time's difference. Besides, when using (\ref{e2}), we can utilize
the cuBLAS library known as highly performant. Section~\ref{ss:pure} falls
into the first category, which intuitively uses (\ref{e1}) and implementes
a native and hand-tuned kernel. Both Section~\ref{ss:cublas} and ~\ref{ss:mix}
use (\ref{e2}). While Section~\ref{ss:cublas} uses cuBLAS library 
functions for computing both matrix inner product and vector norms, 
Section~\ref{ss:mix} implements its own vector norm function by computing the 
product of the matrix comprised by vectors, which turns out to be much 
efficient than cuBLAS's own implementation.


\subsection{Intuitive CUDA implementation}
\label{ss:pure}

One intuitive parallel implementation, which computes the distance 
using (\ref{e1}), is described as Algorithm~\ref{par}. 

\begin{algorithm}[!h]
  \caption{Parallel k-means clustering} \label{par}
  \begin{algorithmic}[1]
    \INPUT $K$: Number of clusters; $N$: number of d-dimensional data points; $p$: data points.
    \Function{par\_k-means}{$p, N, K$} \label{alg:p}
    \State Randomly choose $K$ points as cluster centroids $c[]$
    \While {! termination\_condition}
    \ParFor {i = 1..N}
    \For {j = 1..K}
    \For {dd = 1..d}
    \State $distance += (p(i)(dd) - c(j)(dd))^2$
    \EndFor
    \EndFor
    \State Find the nearest centroid $c_{nearest}$ for $p(i)$
    \State Change membership of $p(i)$ to the cluster with $c_{nearest}$
    \State Accumulate $p(i)$'s coordinates to the cluster's new centroid
    \EndParFor
    \State Compute new $c[]$: divide the accumulated coords by num\_points
    \State Recalculate termination condition
    \EndWhile
    \EndFunction  
  \end{algorithmic}
\end{algorithm}

\vspace{5mm}
\noindent
Suppose $m$ is the number of iterations in a run. The work-depth model for this algorithm is: \\
Work = $O(NKdm)$, Depth = $O(Kdm)$.



\subsection{CUDA implementation with cuBLAS}
\label{ss:cublas}

By using \ref{e2}, we can utilize two functions provided by cuBLAS.
$\norm{\vec{p_i}}$ and $\norm{\vec{c_i}}$ are calculated by \textbf{cublasSgemm()},
and $2\vec{p_i} \cdot \vec{c_i^T}$ by \textbf{cublasSnrm2()}.
Such a parallelization is shown in Algorithm~\ref{par_m}:

\begin{algorithm}[!h]
  \caption{Parallel k-means clustering using cuBLAS} \label{par_m}
  \begin{algorithmic}[1]
    \INPUT $K$: Number of clusters; $N$: number of $d$-dimensional data points; $p$: data points.
    \Function{par\_k-means-matrix}{$p, N, K$} \label{alg:pm}
    \State Randomly choose $K$ points as cluster centroids $c[]$
    \While {! termination\_condition}
    \For {i = 1..N}
    \State $p\_norm^2(i) = \norm{p(i))}^2$
    \EndFor
    \For {j = 1..K}
    \State $c\_norm^2(j) = \norm{(c(j))}^2$
    \EndFor
    \State $pc\_product = 2 p \cdot c^T$
    \ParFor {i = 1..N}
    \For {j = 1..K}
    \State $distance = p\_norm^2(i) + c\_norm^2(j) - pc\_product(i)(j)$
    \EndFor
    \State Find the nearest centroid $c_{nearest}$ for $p(i)$
    \State Change membership of $p(i)$ to the cluster with $c_{nearest}$
    \State Accumulate $p(i)$'s coordinates to the cluster's new centroid
    \EndParFor
    \State Compute new $c[]$: divide the accumulated coords by num\_points
    \State Recalculate termination condition
    \EndWhile
    \EndFunction
  \end{algorithmic}
\end{algorithm}

\vspace{5mm}
\noindent
The work-depth model for this algorithm is: \\
Work = $O(2Ndm + Kdm + NKdm + NKm)$ \\
Depth = $O(D(norm(d))*N*m + (norm(d))*K*m + D(matrix product(NKd))*m + Km + dm)$.

\vspace{3mm}

Since we do not know the internal of cuBLAS implementions, we put place holder
in depth for algorithm~\ref{par_m}\ref{par_vn_m}. 

After the implementation, we find that it is even slower than Algorithm~\ref{par}.
Although \textbf{cublasSgemm()} is highly efficient, \textbf{cublasSnrm2()} 
turns out to be very slow, causing performance degradation.
Therefore, we implemented our own \textbf{norm\_2()} as described in 
Section~\ref{ss:mix}.


\subsection{Mixed cuBLAS with customized functions}
\label{ss:mix}

Given that there are no native cuBLAS function to collectively calculate the
sqaures of norms for each row of a matrix, and that \textbf{cublasSnrm2()} is
not performant enough, we implement our own square of norm function by utilizing
the high performant \textbf{cublasSgemm()}.
Actually, for a matrix A, the norms of each row can be got at the diagonal of 
$A * A^T$. The details are shown in Algorithm~\ref{par_vn_m}. 

\begin{algorithm}[!h]
  \caption{Parallel k-means clustering, using matrix product to compute norm} \label{par_vn_m}
  \begin{algorithmic}[1]
    \INPUT $K$: Number of clusters; $N$: number of $d$-dimensional data points; $p$: data points.
    \Function{par\_k-means-matrix-v2}{$p, N, K$} \label{alg:pm2}
    \State Randomly choose $K$ points as cluster centroids $c[]$
    \While {! termination\_condition}
    \State $p\_norm\_2 = diag(p * p^T)$
    \State $c\_norm\_2 = diag(c * c^T)$
    \State $pc\_product = 2 p \cdot c^T$
    \ParFor {i = 1..N}
    \For {j = 1..K}
    \State $distance = p\_norm\_2(i) + c\_norm\_2(j) - pc\_product(i)(j)$
    \EndFor
    \State Find the nearest centroid $c_{nearest}$ for $p(i)$
    \State Change membership of $p(i)$ to the cluster with $c_{nearest}$
    \State Accumulate $p(i)$'s coordinates to the cluster's new centroid
    \EndParFor
    \State Compute new $c[]$: divide the accumulated coords by num\_points
    \State Recalculate termination condition
    \EndWhile
    \EndFunction
  \end{algorithmic}
\end{algorithm}

\vspace{5mm}
\noindent
The work depth model for this algorithm is: \\
Work = $O(N^2dm + K^2dm + NKdm + NKm)$, \\
Depth = $O(D(matrix product(N^2d))*m + D(matrix product(K^2d)*m) + D(matrix product(NKd))*m+ Km + dm)$.

\documentclass{article}
\usepackage{amsmath, amsfonts}
\usepackage{hyperref}
\usepackage{breakurl}
\usepackage{paralist}
\begin{document}
\title{Project Proposal}
\author{Yige Hu and Zhiting Zhu}
\date{}
\maketitle

\section{Problem Description}
K-means clustering is a unsupervised learning algorithm for solving clustering
problem. Formally, the problem states
as follows~\cite{wiki}: Given a set of data points $\{x_i|i = 1..n\} \subseteq \mathbb{R}^d $, k-means clustering aims to partition the n data points
in to k($\leq n$) sets S = $\{S_1, S_2, ..., S_k\}$ so as to minimize
the within-cluster sum of squared
errors, $$\arg\min_{S}\sum_{i=1}^{k}\sum_{x \in S_i} \parallel x -
\mu(S_i)\parallel$$ where $\mu(S_i)$ is the mean of points in $S_i$. 

\section{Standard Sequential Algorithm}
\begin{compactitem}
\item{Choose the number of clusters, k.}
\item{Randomly generate k points as cluster centers.}
\item{Assign each point to the nearest cluster center.}
\item{Recompute the new cluster centers.}
\item{Repeat the previous two steps until some convergence criterion
  is met.}
\end{compactitem}

\section{Plan}
We plan to use GPU and CUDA to implement the parallel version of
k-means algorithm. For GPU, we will use NVIDIA GPU Tesla K20c to test
and bench mark our implementation. We will also compare our
implementation with the existing implementation we find on the
web\cite{serban-kmeans, gpuminer}.

\bibliographystyle{acm}
\bibliography{bibliography}  
\end{document}
